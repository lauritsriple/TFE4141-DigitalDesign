\section{Verification plan and verification done}
%
\subsection{Verification plan}
The verification plan for the system consists of creating a type I test bench
\cite[p.244]{pedroni} for the monPro algorithm, the inner module in the Montgomery algorithm,
and checking the results manually against the output from the Python script (\cref{app:python}).
This allows for a fairly rigorous verification without the need for writing a fully
automated test bench (type III/IV). 

For the complete system a modified variant of the example test bench given along with the 
assignment is used. The test bench is modified to accept two user defined parameters 
supplied with the keys and message. The advantage to this form of test bench is that
the output is automatically verified and thus leave no room for human error. The backbone
of the automatic test bench for the complete system are \emph{assertions}. The assertions
guarantee that for every message passed to the system the correct encrypted message is outputted.
The code with assertions can be seen in \cref{app:test_bench}.

Since the software used for simulation, Vivado, does not support timing simulation in VHDL
\cite{vivado_timing_sim}, the only functional simulations have been completed. 
This is only a minor issue as once the system passes functional simulations, 
it can be tested on a physical FPGA. The RSA system should also pass a pre-made test on an actual
FPGA. In this project it was possible to log on to a remote server, connected to a FPGA, that could
compile and run the system on a real target. 

Both of these test benches are given in \cref{app:test_bench}. It might have been more
appropriate to make test benches for every process, but also more time consuming. For this
project, a fairly small project, the process of making all the test benches, since most of
the processes are not identical would be rather time consuming and would have further 
increased the overall complexity of the system. 

The main verification metric for the system is pass rate of the individual test benches, 
both in simulation and on the FPGA. Due to the size of the project (less than 1000 lines of 
design sources) it is possible to say with confidence that both the code coverage and the 
functional coverage is close to, if not 100 \%. 
%
\subsection{Bring up test strategy}
The bring up strategy used to get this system up and running was a bottom-up strategy. The first module
implemented was the inner monPro algorithm. Once that was verified with the testbench with different
stimulis, the top level modExp module was implemented. At this point it is possible to know that
problems with the module at the top, cannot propagate down to the underlying modules, so if there are
any errors here, it is purely because somthing is wrong in the top module, since the bottom module is
already tested.
