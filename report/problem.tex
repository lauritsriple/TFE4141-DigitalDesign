\section{Problem Description and Analysis}
The problem in this paper is to make an algorithm to encrypt and decrypt messages using the
RSA algorithm and implement it on an fpga. The RSA algorithm was invented by Rivest, Shamir
and Adleman in 1977. The algorithm consists of 5 parameters; $n$,$p$,$q$,$e$ and $d$. 
$p$ and $q$ are to random large prime numbers, and $n$ can be calculated by multiplying 
them together, $n=pq$. $e$ is the public exponent and has to be in the range $1<e<\phi(n)$ 
such that $gcd(e,\phi(n))=1$. $\phi(n)$ is Euler's totient function of n, given by
\begin{equation}
    \phi(n)=(p-1)(q-1)
\end{equation}
The private exponent, $d$, is obtained by inverting $e$ $modulo$ $\phi(n)$, i.e
\begin{equation}
    d=e^{-1}\mod{\phi(n)}
\end{equation}
The encryption operation is performed by computing
\begin{equation}
    C=M^e\mod{n}
\end{equation}
where M is the plaintext such that $0<=M<=n$. The number C is the ciphertext from 
which the plaintext M can be computed using
\begin{equation}
    M=C^d\mod{n}
\end{equation}
The RSA algorithm can be used to produce signatures for electronic documents as well
as used to send encrypted messages. 

\subsection{Describe the problem you are trying to solve}
The problem with implementing this in hardware on an FPGA is that the exponential multiplication
$M^n$ can be very computational heavy, depending on the choosen algorithm. With a keysize of 55,
the easiest method is to just muliply $n$ with $n$ 55 times. This requires 54 multiplication
operations, and would take a very long time with large numbers. This is a brute-force approach
and has complexity $O(n)$. Another approach is using repeated squaring. With $n$ written as binary a
number,$n=55=0x37=0b00110111$, it can be written as a sum of squared exponents of 2, 
$E=0\cdot2^7 + 0\cdot2^6 + 1\cdot2^5 + 1\cdot2^4 + 1\cdot2^2 + 1\cdot2^1 + 1\cdot2^0=55$. The exponential multiplication can now be written as
\begin{equation}
M^{55}=M\exp(0\cdot2^7 + 0\cdot2^6 + 1\cdot2^5 + 1\cdot2^4 + 1\cdot2^2 + 1\cdot2^1 + 1\cdot2^0) 
= M^{2^5} + M^{2^4} + M^{2^2} + M^{2^1} + M^{2^0} 
\end{equation}
From this it is easy to see a patt
\subsection{What are the main requirements?}
\subsection{What do you need to investigate further before proposing a solution?}
\subsection{Discuss/Analyze/Conclude}


\subsection{KEYWORDS}
Descirbe and FPGA
Descirbe the rsa algorithm (history and math)
Many ways to implement it on hardware
Describe montgogemry in math and some bla,bla about the algorithm
What is the strengths of doing this.

Main requirements. How fast do we need it to be? How many bits etc.

Investigate further. pytyhon model. reading other papers etc.
